\documentclass{article}

\usepackage[T1]{fontenc} 
\usepackage{ngerman}
\usepackage[a4paper,lmargin={2.5cm},rmargin={2.5cm},tmargin={2cm},bmargin={2.5cm}]{geometry}
                   
\begin{document}
\title{Softcomputing - Begriffsdefinition}
\author{Sebastian Sch�tteler}
\maketitle

%Eine Allgemeine Definition zu Softcomputing
\section{Softcomputing als Sammelbegriff}
Der im Jahre 1991 von Lofti A. Zadeh ins Leben gerufene Begriff ''Softcomputing'' befasst sich mit numerischen Verfahren zur Ermittlung von N�herungsl�sungen. Oft wird ebenfalls der Begriff ''Naturanaloge Verfahren'' verwendet. Dieser Begriff kam durch die Art und Weise zustande, wie die numerischen Verfahren arbeiten - sie lernen durch Erfahrung wie es in der Natur �blich ist. Softcomputing kann als Sammelbegriff f�r die vier folgenden Themen verstanden werden: K�nstliche Neuronale Netze, Fuzzy-Logik, Evolution�re Algorithmen, Chaos-Theorie. Auf diese Begriffe wird nun in den folgenden Punkten genauer eingegangen.

\begin{itemize}
	\item Neuronale Netze \\
	Neuronale Netze stellen eine Analogie des Neuronennetz des menschlichen Gehirns dar. Ein (k�nstliches) neuronales Netz besteht aus mehreren Neuronen. Neuronen, die Informationen aus der Umwelt aufnehmen und Neuronen, die Informationen in modifizierter Form an die Umwelt weitergeben. Diese Neuronen sind entsprechend miteinander verbunden.  

	\item Fuzzy-Logik
	\item Evolution�re Algorithmen
	\item Chaoes-Theorie
\end{itemize}

%Alternativen zu Softcomputing und deren Bedeutungsschwerpunkte
\section{Alternativen zu Softcomputing}

%Abgrenzung zu anderen Themengebieten und deren Bedeutungsschwerpunkte
\section{Abgrenzung zu anderen Themengebieten}
Softcomputing steht in Abgrenzung zu ''Hard Computing'',... 

%Auf der Zweiten Seite befindet sich ein Literaturverzeichnis mit circa 25 Quellen
\newpage
\begin{thebibliography}{}
%Anfang
\bibitem{book00}
  Uwe L�mmel \& J�rgen Cleve,
  \emph{K�nstliche Intelligenz},
  Hanser Verlag, M�nchen,
  3., neu bearbeitete Auflage,
  2008.
	
	\bibitem{book01}
  Wolfram-Manfred Lippe,
  \emph{Softcomputing},
  Springer Verlag, Berlin Heidelberg,
  1. Auflage,
  2006.
	
	\bibitem{url00}
	Volker Nissen,  
	\emph{Soft Computing}, 
	http://www.\mbox{enzyklopaedie-der-wirtschaftsinformatik.de}\\/lexikon/technologien-methoden/KI-und-Softcomputing/Softcomputing,
  2008, 
  Online, 
	Zugriff am 12.10.2015.
	
		\bibitem{url01}
	Gabler Wirtschaftslexikon,  
	\emph{Soft Computing}, 
	http://wirtschaftslexikon.gabler.de/Definition/soft-computing.html,
  Keine Datumsangabe, 
  Online, 
	Zugriff am 12.10.2015.
	
	%Neuronale Netze
	\bibitem{url01}
	G�nter Daniel Ray,  
	\emph{Einleitung in neuronale Netze}, 
	http://www.neuronalesnetz.de/einleitung.html,
  Keine Datumsangabe, 
  Online, 
	Zugriff am 12.10.2015.
	
	%Fuzzy-Logik
	%Evolution�re Algorithmen
	%Chaos-Theorie
\end{thebibliography}

\end{document}
