\documentclass{article}

\usepackage[T1]{fontenc} 
\usepackage{ngerman}
\usepackage[a5paper,lmargin={1.5cm},rmargin={1.5cm},tmargin={1cm},bmargin={1.5cm}]{geometry}
                   
\begin{document}
\title{Softcomputing}
\author{Sebastian Sch�tteler}

\maketitle
\small
%Eine Allgemeine Definition zu Softcomputing
\section{Softcomputing}
\begin{itemize}
	\item Approximation von Optimall�sungen mittels Lernalgorithmen. 
	\item Alternativbegriff: ''Naturanaloge Verfahren''.
	\item Sammelbegriff f�r: K�nstliche neuronale Netze, Fuzzy-Logik, evolution�re Algorithmen und Methoden der Chaostheorie.


\begin{itemize}
\item \textbf{K�nstliche neuronale Netze} \\
K�nstliche neuronale Netze stellen eine Analogie des Neuronennetz des menschlichen Gehirns dar.

\item \textbf{Fuzzy-Logik} \\
Bei der Fuzzy-Logik handelt es sich um eine Verallgemeinerung der zweiwertigen Booleschen Algebra. 
		
\item \textbf{Evolution�re Algorithmen} \\
Bei evolution�ren Algorithmen handelt es sich um eine Verfahrensklasse, mit der Probleme nach dem Vorbild der biologischen Evolution gel�st werden. 
	
\item \textbf{Methoden der Chaos-Theorie} \\
Die Chaos-Theorie stammt urspr�nglich aus der Physik und und modelliert das Verhalten komplexer r�ckgekoppelter Systeme.
 
\end{itemize}
\end{itemize}
%Abgrenzung zu anderen Themengebieten und deren Bedeutungsschwerpunkt
\section{\textbf{Abgrenzung zu anderen Themengebieten}}
Abzugrenzen ist der Begriff ''Softcomputing'' gegen�ber dem Begriff ''Hardcomputing''.
\begin{itemize}
	\item Hardcomputing: deterministisch, ben�tigt vordefiniertes Programm sowie genaue Angaben.
	\item Softcomputing: erlaubt vage, ungenaue, unvollst�ndige sowie nur partiell wahre Informationen. Programm ''entwickelt'' sich von selbst.
\end{itemize}
\end{document}
