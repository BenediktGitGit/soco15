\documentclass{article}

\usepackage[T1]{fontenc} 
\usepackage{ngerman}
\usepackage[a4paper,lmargin={2.5cm},rmargin={2.5cm},tmargin={2cm},bmargin={2.5cm}]{geometry}
                   
\begin{document}
\title{Soft-Computing - Begriffsdefinition}
\author{Benedikt Hofrichter}
\maketitle

%Eine Allgemeine Definition zu Softcomputing
\section{Softcomputing als Sammelbegriff}
Der Begriff ''Soft-Computing'' ist ein Begriff den erstmals Lotfi A. Zadeh pr�gte. Rechenverfahren, die sich unter diesem Begriff zusammenfassen lassen orientieren sich an der nat�rlichen Informationsverarbeitung, also ganz nach dem Prinzip, ''die Natur macht keine Spr�nge'' und ''gut Ding braucht Weile''. Solche Verfahren generieren gen�herte (approximative) L�sungen f�r ein Problem, die ausreichend sind, um Aussagen �ber dieses zu erm�glichen. Es hat sich die letzten Jahrzente herauskristallisiert, das im Wesentlichen vier Kategorien die Grundpfeiler des Softcomputings bilden: 

\begin{itemize}
	\item Fuzzy-Logik \\
	Fuzzy-Logik ist die mathematische Disziplin, die das logische Schlie"sen der boolschen Algebra und das 			mathematisch nicht korrekt Fassbare zusammenf�hrt. Die typischen boolschen Spr�nge zwischen 0 und 1 werden 		durch das Intervall [0,1] ersetzt. Somit ist die boolesche Logik ein Sonderfall der Fuzzy Logik. Die
	Fuzzifikation ist der Arbeitsschritt, der vage (unscharfe) Aussagen in Fuzzy-Mengen interpretiert. 
	Hierdurch wird ein Fuzzy-Schlie"sverfahren erm�glicht.

	\item K�nstliche neuronale Netze \\
	K�nstliche Neuronale Netze (ANNs) sind dem menschlichen Gehirn nachempfunden und versuchen die Funktionen 
	dieses zu imitieren. Hierbei spielen Neurone als Knotenpunkte in einem ANN die Hauptrolle. Diese sind f�r 
	die Informationsverarbeitung zust�ndig und k�nnen als Sender (Emitter) und Empf�nger (Tansmitter)
	fungieren. Wie ein menschliches Hirn kann ein neuronales Netz trainiert werden.
		
	\item Chaos-Theorie \\
	Die Chaosforschung wird schon �ber ein Jhd. betrieben und entspringt der mathematischen Physik. Im 
	Wesentlichen besch�ftigt Sie sich mit der Ordnung im Chaos. Anwendungen im Softcomputing finden sich 
	z.B. im Bereich ''Prediktive Systeme'', wenn es um die Vorhersage auf Grundlage von schwach strukturierten
	Daten / Daten mit fragw�rdigem Zusammenh�ngen geht (Aktienkurse, Verkehrsaufkommen etc.).
	
	\item Evolution�re Algorithmen \\
	Evolution�re Algorithmen kennzeichnen eine Klasse von stochastisch, metaheuristischen
	Optimierungsverfahren. Die Funktionalit�t orientiert sich am nat�rlichen Evolutionsprozess. Begriffe wie
	Erbgut, Chromosome, Allele werden hier wie auch in der Genetik verwendet. Die genetischen 
	Algorithmen stellen dabei die wichtigste Unterklasse dar.  
	
\end{itemize}


%Abgrenzung zu anderen Themengebieten und deren Bedeutungsschwerpunkte
\section{Abgrenzung zu anderen Themengebieten}
Soft-Computing grenzt sich zu Rechenverfahren mit exakten L�sungen ab. 

%Auf der Zweiten Seite befindet sich ein Literaturverzeichnis mit circa 25 Quellen
\newpage
\begin{thebibliography}{}
\renewcommand\refname{Reference}
%Anfang
\bibitem{book00}
  Uwe L�mmel \& J�rgen Cleve,
  \emph{K�nstliche Intelligenz},
  Hanser Verlag, M�nchen,
  3., neu bearbeitete Auflage,
  2008.
	\bibitem{book01}
  Wolfram-Manfred Lippe,
  \emph{Softcomputing},
  Springer Verlag, Berlin Heidelberg,
  1. Auflage,
  2006.
	
	\bibitem{url00}
	Volker Nissen,  
	\emph{Soft Computing}, 
	http://www.\mbox{enzyklopaedie-der-wirtschaftsinformatik.de}\\/lexikon/technologien-methoden/KI-und-Softcomputing/Softcomputing,
  2008, 
  Online, 
	Zugriff am 12.10.2015.
	
	\bibitem{url01}
	Gabler Wirtschaftslexikon,  
	\emph{Soft Computing}, 
	http://wirtschaftslexikon.gabler.de/Definition/soft-computing.html,
  Keine Datumsangabe, 
  Online, 
	Zugriff am 12.10.2015.
	
	%Neuronale Netze
	\bibitem{url02}
	G�nter Daniel Ray,  
	\emph{Einleitung in neuronale Netze}, 
	http://www.neuronalesnetz.de/einleitung.html,
  Keine Datumsangabe, 
  Online, 
	Zugriff am 12.10.2015.
	
	%Fuzzy-Logik
	\bibitem{url05}
	Wikipedia,  
	\emph{Fuzzylogik}, 
	https://de.wikipedia.org/wiki/Fuzzylogik,
  16.09.2015, 
  Online, 
	Zugriff am 26.09.2015.
	
	%Evolution�re Algorithmen
	\bibitem{url03}
	Wikipedia,  
	\emph{Evolution�rer Algorithmus}, 
	https://de.wikipedia.org/wiki/Evolution�rer\_Algorithmus,
  16.09.2015, 
  Online, 
	Zugriff am 12.10.2015.
	
	
	\bibitem{book02}
	Ingrid Gerdes, Frank Klawonn \& Rudolf Kruse,  
	\emph{Evolution�re Algorithmen}, 
	Friedr. Vieweg \& Sohn Verlag, Wiesbaden,
	1. Auflage,
  2004.
	
	\bibitem{url04}
	Karsten Weicker,  
	\emph{Evolution�re Algorithmen}, 
	http://www.imn.htwk\mbox{-}leipzig.de\~weicker/publications/sc-treff\_ea.pdf,
  Keine Datumsangabe, 
  Online, 
	Zugriff am 12.10.2015.
\end{thebibliography}

	%Chaos-Theorie

\end{document}
