\documentclass[11pt,a4paper]{scrreprt}

\usepackage[ngerman]{babel} 
\usepackage[T1]{fontenc}           
\usepackage[utf8]{inputenc}  
\usepackage[a4paper,lmargin={2.25cm},rmargin={2.0cm},tmargin={2.5cm},bmargin={2.0cm}]{geometry}
\usepackage{graphicx}

\usepackage[fixlanguage]{babelbib} 
\selectbiblanguage{german} 

\pdfinfo
{
	/Title		(Extrapolation von Zeitreihen mit Hilfe von künstlichen neuronalen Netzen am Beispiel von Börsenprognosen)
	/Subject	(Eine Anwendung im Fach Softcomputing – Teilgebiet neuronale Netze)
	/Author		(Sebastian Schötteler \& Benedikt Hofrichter)
}

\begin{document}


\title
{
 {\bf Extrapolation von Zeitreihen mit Hilfe von künstlichen neuronalen Netzen am Beispiel von Börsenprognosen}
}

\author
{
	Sebastian Schötteler \& Benedikt Hofrichter \\
	Matrikelnummer 2429289 \& Matrikelnummer 2272198 \\\\ 
	Technische Hochschule Nürnberg Georg Simon Ohm \\
}


\maketitle

\tableofcontents


\chapter{Einleitung}
	\section{Motivation}
Die Untersuchung  und Extrapolation von Zeitreihen ist ein bedeutendes Thema in zahlreichen Gebieten. Typische Anwendungsbereiche sind dabei Prognose von Wetterdaten, von Therapieverläufen in der  Medizin und Psychologie, von Arbeitslosenzahlen auf dem Arbeitsmarkt sowie Börsenkursen. Um eine Zeitreihe möglichst genau zu extrapolieren, wird auf mehreren Hilfsmitteln zugegriffen. Einer dieser Hilfsmittel sind künstliche neuronale Netze (Abgekürzt: KNN). Bei künstlichen neuronale Netzen handelt es sich um ein in sich geschlossenes System von Neuronen, die die Eingabe weiterverarbeiten und das Ergebnis an weitere Neuronen weiterleiten.  

	\section{Ziel dieser Arbeit}
Die Funktionsweise und Effektivität von KNN bei der Extrapolation von Zeitreihen soll anhand einer Anwendung, die den Boersenkurs des DAX für die nächsten Börsentage prognostiziert, ermittelt und anschließend demonstriert werden.

\chapter{Konzeption} %Beide
	\section{Konzeption der Anwendung} %Benedikt
		\subsection{Funktionalitäten der Anwendung} %Benedikt
		\subsection{Mockup der Anwendung} %Benedikt
	\section{Konzeption des künstlichen neuronalen Netzes} %Sebastian
		\subsection{Typ des künstlichen neuronalen Netzes} %Sebastian
		\subsection{Architektur des künstlichen neuronalen Netzes} %Sebastian
		\subsection{Lernverfahren des künstlichen neuronalen Netzes} %Sebastian
	\section{Analyse von Frameworks} %Benedikt
		\subsection{SNNS} %Benedikt
	  \subsection{JavaNNS}  %Benedikt
		\subsection{Neuroph} %Benedikt
	\section{Wahl des geeignetsten Frameworks} %Benedikt
	
\chapter{Umsetzung} %Beide
	\section{Erstellung des künstlichen neuronalen Netzes} %Sebastian
		\subsection{Wahl der Topologie} %Sebastian
		\subsection{Wahl der Transferfunktion} %Sebastian
		\subsection{Wahl der Lernregel} %Sebastian
	\section{Überführung des künstlichen neuronalen Netzes in einer Anwendung}
	\section{Anpassen der Anwendung} %Benedikt

\chapter{Beschreibung der Anwendung} %Benedikt
	\section{Elemente der GUI} %Benedikt
	\section{Architektur der Anwendung} %Benedikt
	\section{Zusammenspiel mit dem Framework} %Benedikt

\chapter{Fazit} %Beide
Das prognostizieren von Börsenkursen mittels künstlichen neuronalen Netzen ist möglich.

\chapter{Appendix}
	\section{Abbildungsverzeichnis}
	\section{Tabellenverzeichnis}
	\section{Literaturverzeichnis}

\end{document}