\documentclass[11pt,a4paper]{scrreprt}

\usepackage[german]{babel}    % Deutsche Sprache in automatisch generiertem
\usepackage{latexsym}         % Fuer recht seltene Zeichen
\usepackage[utf8]{inputenc}   % =E4 =F6 =FC =DF; danach  geht auch das ß richtig
\usepackage{caption}          % Figure-Captions formatieren
\usepackage{sectsty}          % Section headings formatieren
\usepackage[fixlanguage]{babelbib} %Korrektur
\usepackage[a4paper,lmargin={2.5cm},rmargin={2.5cm},tmargin={2cm},bmargin={2.2cm}]{geometry} %Seitenformatierung
\usepackage{graphicx}
\usepackage{tocstyle}

%Informationen zum Text
\pdfinfo
{
	/Title		(Extrapolation von Zeitreihen mit Hilfe von künstlichen neuronalen Netzen am Beispiel von Börsenprognosen)
	/Subject	(Eine Anwendung im Fach Softcomputing – Teilgebiet neuronale Netze)
	/Author		(Sebastian Schötteler \& Benedikt Hofrichter)
}

%Sonstige Einstellungen
\selectbiblanguage{german} %Korrektursprache

%Ab hier beginnt das Dokument
\begin{document}

%Titel versehen
\title
{
\bf Extrapolation von Zeitreihen mit Hilfe von künstlichen neuronalen Netzen am Beispiel von Börsenprognosen
\author
{
	Sebastian Schötteler \& Benedikt Hofrichter \\
	Matrikelnummer 2429289 \& Matrikelnummer 2272198 \\\\ 
	Technische Hochschule Nürnberg Georg Simon Ohm \\
}
%\date{\today}
}

\maketitle


\tableofcontents

\chapter{Einleitung}
 \section{Motivation}
 \section{Ziel dieser Arbeit}

\chapter{Konzeption}
	\section{Konzeption der Java-Anwendung}
		\subsection{Funktionalitäten der Anwendung}
		\subsection{Mockup der Anwendung}
	\section{Konzeption des künstlichen neuronalen Netzes}
		\subsection{Klasse des künstlichen neuronalen Netzes}
		\subsection{Topologie des künstlichen neuronalen Netzes}
		\subsection{Lernverfahren des künstlichen neuronalen Netzes}
	\section{Analyse geeigneter Frameworks zur Erstellung künstlicher neuronaler Netze}
		\subsection{SNNS}
	  \subsection{JavaNNS} 
		\subsection{Neuroph}
	\section{Wahl des geeignetsten Frameworks}
	
\chapter{Umsetzung}
	\section{Erstellung künstlicher neuronaler Netze}
	 \subsection{4-09-1 – Layer mit Sigmoider Funktion}
	 \subsection{4-09-1 – Layer mit Tanh Funktion}
	 \subsection{4-21-1 – Layer mit Sigmoider Funktion}
	 \subsection{4-21-1 – Layer mit Tanh Funktion}
  \subsection{Validierung der künstlichen neuronalen Netze}
	\section{Überführung der künstlichen neuronalen Netze in einer Java-Anwendung}
	\section{Anpassen der Java-Anwendung}

\chapter{Beschreibung der Anwendung}
 \section{Elemente der Anwendung}
 \section{Zusammenspiel mit dem Framework}

\chapter{Fazit}

\chapter{Appendix}
\section{Abbildungsverzeichnis}
\section{Tabellenverzeichnis}
\section{Literaturverzeichnis}

\end{document}