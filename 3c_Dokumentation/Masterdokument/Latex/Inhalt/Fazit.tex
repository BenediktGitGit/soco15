\chapter{Zusammenfassung und Fazit}
\label{cha:Zusammenfassung und Fazit}

Die KNN dieser Seminararbeit liefern zwar gute Approximationen, jedoch sind diese für praktische Zwecke noch nicht ausreichend. Dafür sind primär zwei Ursachen verantwortlich. Zunächst ist zu berücksichtigen, dass die KNN in dieser Anwendung ein abgeschottetes System bilden. Das bedeutet, dass diese nicht in der Lage sind, auf einschneidende Ereignisse (wie z.B. Terroranschläge) angemessen zu reagieren, obwohl solche einen großen Einfluss auf den Börsenkurs haben können. Eine Erweiterung um diese Eingaben wäre prinzipiell möglich, jedoch sehr aufwändig. Ein weiteres Manko der in dieser Seminararbeit erstellten KNN ist das Fehlen von nichtlinearen Zusammenhängen. Es wurden lediglich die letzten vier Börsenkurse zur Prognose des darauffolgenden Kurses verwendet. Erweiterungen wie z.B. durch den Leitzins oder Kurse anderer Börsen als Input würden die Prognosefähigkeit wahrscheinlich stark steigern, denn genau hier erweisen sich KNN als besonders effektiv.
 
Auf dem Markt befinden sich bereits zahlreiche Anbieter von sehr ausgefeilten Anwendungen auf Basis von KNN, die Börsenkurse prognostizieren. Diese liefern tatsächlich recht genaue Ergebnisse, die auch in der Praxis vom Nutzen sein können. Der Preis zur Nutzung dieser Anwendungen ist jedoch recht hoch. So stellt sich die Frage, ob der Nutzen tatsächlich höher ist als die Kosten zu Nutzung eines KNN dieser Anbieter.

Zusammenfassend kann festgehalten werden, dass die Prognose von Börsenkursen mittels eines KNN möglich ist, jedoch mit sehr viel Aufwand verbinden ist, wenn man praxistaugliche Ergebnisse erzielen möchte. Auch sollte ein KNN nie als alleiniges Prognoseinstrument, sondern immer nur als Ergänzung zu anderen Prognoseinstrumenten eingesetzt werden.
