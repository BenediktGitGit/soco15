\chapter{Einleitung}
\label{chapter:Einleitung}

\section{Motivation}
\label{section:Motivation}
Die Untersuchung und Extrapolation von Zeitreihen ist ein bedeutendes Thema in zahlreichen Gebieten. Typische Anwendungsbereiche sind zum Beispiel die Prognose von Wetterdaten, von Therapieverläufen in der  Medizin, von Arbeitslosenzahlen auf dem Arbeitsmarkt sowie von Börsenkursen. Um eine Zeitreihe möglichst genau zu extrapolieren, wird auf mehrere Hilfsmittel zurückgegriffen. Eins dieser Hilfsmittel können künstliche neuronale Netze (\acs{knn}) sein. 

Bei \acs{knn} handelt es sich um Netzwerke mit künstlichen Neuronen als Knoten, die mittels gerichteter Verbindungen Eingaben einlesen, weiterverarbeiten und die daraus resultierenden Ergebnisse an weitere Neuronen weiterleiten oder als Ergebnis ausgeben. Bei der Terminologie von \acs{knn} wird bewusst auf Begriffe der Biologie zurückgegriffen, da \acs{knn} das biologische Gehirn als Vorbild nutzen und dessen Herangehensweise auf analoger Weise umzusetzen zu versuchen. Man nennt das Verfahren dieser Netze aus diesem Grunde auch \textit{naturanaloge Verfahren}.

Warum sind diese Netze nun so interessant für Prognosen? Das Erstellen von zum Beispiel Börsenprognosen basiert in der Regel auf Auswertungen von Informationen verschiedener Quellen. Die Art von Auswertungen, wie Börsenexperten sie vornehmen, ist weder vollständig formalisierbar noch besonders exakt, da uneinheitlich und in weiten Zügen intuitiv. Besonders schwer ist hier das Ermitteln von \textit{nichtlinearen Zusammenhängen}. Ein \acs{knn} ist jedoch in der Lage, diese Zusammenhänge zu finden  und diese objektiv und vorurteilsfrei zu bewerten. Somit sind diese prinzipiell in der Lage, jedes beliebige Muster in jedem beliebigen Markt zu erkennen - auch solche, die noch nie zuvor von irgend jemand entdeckt wurden.

Ob und wie gut \acs{knn} zur Prognose geeignet sind, ist pauschal nicht zu beantworten. In manchen Gebieten mag die Prognosefähigkeit durchaus ausreichen. Je höher die geforderte Genauigkeit jedoch wird, desto diskutabler wird ein Einsatz von \acs{knn}. Eine typische Grauzone ist hier wieder die Prognose von Börsenkursen. Während Befürworter auf die Eigenschaft von \acs{knn} hinweisen, nichtlineare Muster zu erkennen und entsprechend zu behandeln, argumentieren Kritiker, dass ein System, das dem menschlichen Lernen nachempfunden wurde, die gleichen Fehler machen wird wie der Mensch. Generell ist jedoch zu sagen, dass die Prognosequalität von \acs{knn} über die Jahre stets angestiegen ist.

\section{Ziel der Arbeit}
\label{section:Ziel der Arbeit}
In dieser Seminararbeit sollen \acs{knn} erschaffen werden, die in der Lage sind, Börsenkurse zu prognostizieren. Konkret sollen jeweils ein \acs{knn} zur Prognose des Kurses vom Deutschen Aktienindex, vom Nikkei sowie vom Dow Jones konzeptioniert und umgesetzt werden. Diese \acs{knn} sollen anschließend in einer Webanwendung überführt werden. Diese soll die Prognosefähigkeit der \acs{knn} visualisieren und Vergleiche zwischen einzelnen Prognosen ermöglichen. In dieser Seminararbeit liegt der Fokus auf das Erlangen eines Grundverständnisses über \acs{knn} und nicht auf das komplette Ausreizen der Prognosefähigkeit von \acs{knn}. Trotzdem spielt die Prognosequalität der erstellten \acs{knn} eine wichtige Rolle in dieser Seminararbeit.

%\section{Vorgehensweise}
%\label{section:Vorgehensweise}




