\documentclass{article}

\usepackage[T1]{fontenc} 
\usepackage{ngerman}
\usepackage[a4paper,lmargin={2.5cm},rmargin={2.5cm},tmargin={2cm},bmargin={2.5cm}]{geometry}
                   
\begin{document}
\title{Softcomputing\\Bibliografie zu neuronalen Netzen}
\author{Sebastian Sch�tteler (Matrikelnummer: 2429289)\\Benedikt Hofrichter (Matrikelnummer: 2272198)}
\maketitle


\section{Grundprinzip}
\begin{thebibliography}{}

  \bibitem{00}
  Uwe L�mmel \& J�rgen Cleve,
  \emph{K�nstliche Intelligenz},
  Hanser Verlag, M�nchen,
  3., neu bearbeitete Auflage,
  2008.
	
	\bibitem{01}
  Peter Nachbar,
  \emph{Entwurf Robuster neuronaler Netze},
  Verlag Shaker, Aachen,
  1. Auflage,
  1995.
	
	\bibitem{02}
  Matthias Haun,
  \emph{Einf�hrung in die rechnerbasierte Simulation Artifiziellen Lebens},
  Expert Verlag, Renningen,
  1. Auflage,
  2004.
	
	\bibitem{03}
  Ra�l Rochas,
  \emph{Theorie der neuronalen Netze},
  Springer Verlag, Berlin,
  4. Auflage,
  1996.
	
	\bibitem{04}
  Sybille Kr�mer,
  \emph{Geist-Gehirn-K�nstliche Intelligenz},
  de Gruyter Verlag, Berlin,
  1. Auflage,
  1994.
	
	\bibitem{05}
  Helge Ritter, Thomas Martinez \& Klaus Schulten,
  \emph{Neuronale Netze},
  Addison-Wesley, M�nchen
  2. �berarbeitete Auflage,
  1991.
	
\end{thebibliography}
\section{Erweiterungen und Varianten}
\begin{thebibliography}{}

  \bibitem{10}
  H.-H Bothe,
  \emph{Neuro-Fuzzy-Methoden},
  Springer Verlag, Berlin,
  1. Auflage,
  1998.
	
	\bibitem{11}
  Detlef Nauck, Frank Klawonn \& Rudolf Kruse,
  \emph{Neuronale Netze und Fuzzy-Systeme},
  Vieweg Verlag, Braunschweig/Wiesbaden,
  2. Auflage,
  1996.
	
	\bibitem{12}
  Ronald R. Yager \& Lotfi A. Zadeh,
  \emph{Fuzzy Sets, Neural Networks},
  Van Nostrand Reinhold, New York,
  1. Auflage,
  1994.
	
	\bibitem{13}
  Ewa J. D�nitz,
  \emph{Effizientere Szenariotechnik durch teilautomatische Generierung von Konsistenzmatrizen},
  Universit�t Bremen (Dissertation), Bremen,
  1. Auflage,
  2008.
	
		\bibitem{14}
  Ke-Lin Du,
  \emph{Neural networks and statistical learning},
  Swami M.N.S., London,
  1. Auflage,
  2014.
	
		\bibitem{15}
   Akhmet Marad,
  \emph{Neural Networks with Discontinuous/Impact Activations}.

\end{thebibliography}
%\section{Struktur - Einteilung in Teilgebiete}
%\begin{thebibliography}{}
%\end{thebibliography}
\section{Anwendungsbereiche}
\begin{thebibliography}{}

  \bibitem{30}
  Christian Eurich,
  \emph{Objektlokalisation mit neuronalen Netzen},
  Verlag Harri Deutsch, Frankfurt am Main,
  1. Auflage,
  1995.

	 \bibitem{31}
  Heike Speckmann,
  \emph{Neuronale Netze im praktischen Einsatz},
  Vieweg Verlag, Wiesbaden
  1. Auflage,
  1996.
		
	 \bibitem{32}
  Patricia S. Churchland \& Terrence Sejnowski,
  \emph{Grundlagen zur Neuroinformatik und Neurobiologie},
  Vieweg Verlag, Wiesbaden
  1. Auflage,
  1997.
	
	 \bibitem{33}
  Michael Beuschel,
  \emph{Neuronale Netze zur Diagnose und Tilgung von Drehmomentschwingungen am Verbrennungsmotor},
  Technische Universit�t M�nchen (Dissertation), M�nchen,
  1. Auflage,
  2000.
	
	 \bibitem{34}
  Sven F. Crone,
  \emph{Neuronale Netze zur Prognose und Disposition im Handel},
  Universit�t Hamburg (Dissertation) ,Hamburg
  1. Auflage,
  2008.
	
	\bibitem{35}
  Nicole Ramona B�chel,
  \emph{Identifizierung von Hefen durch Fourier-transform Infrarotspektroskopie und k�nstliche neuronale Netze},
  Technische Universit�t M�nchen (Dissertation), M�nchen
  1. Auflage,
  2009.
	
\end{thebibliography}
%\section{Forschungsthemen}
%\begin{thebibliography}{}
%\end{thebibliography}
%\section{Querverbindungen}
%\begin{thebibliography}{}
%\end{thebibliography}
\end{document}
